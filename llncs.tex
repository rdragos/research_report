% This is LLNCS.DEM the demonstration file of
% the LaTeX macro package from Springer-Verlag
% for Lecture Notes in Computer Science,
% version 2.4 for LaTeX2e as of 16. April 2010
%
\documentclass{llncs}
%
\usepackage{makeidx}  % allows for indexgeneration
\usepackage{llncsdoc}
\usepackage{algorithmic}
\usepackage{amssymb}
\usepackage{graphicx}
\usepackage{wrapfig}
\usepackage{cite}
\usepackage{amsmath}
%
\begin{document}
\pagestyle{empty}
%
%
\title{Raport de cercetare pentru UROP}
%
\titlerunning{Research report for UROP course}  % abbreviated title (for running head)
%                                     also used for the TOC unless
%                                     \toctitle is used
%
\author{Drago\c{s} Alin Rotaru}
\authorrunning{D.A.Rotaru} % abbreviated author list (for running head)
%
%%%% list of authors for the TOC (use if author list has to be modified)
%\tocauthor{}

\institute{University of Bucharest, Romania\\
\email{r.dragos0@gmail.com
}
}


\maketitle              % typeset the title of the contribution

\begin{abstract}
  None
\keywords{none}
\end{abstract}
%
%----------------------------------------------------------------
%----------------------------------------------------------------
%----------------------------------------------------------------
%----------------------------------------------------------------
%----------------------------------------------------------------
\section{Introducere}
\label{sec:intro}

\section{Scheme de partajare}
\label{sec:encryption}

\subsection{Constructii existente}
\subsection{Clasificare}

\section{Criptare vs scheme de partajare}
\label{sec:crypt_vs_sharing}

\section{Rezultate obtinute}

\subsection{Arhitectura sistemului}
\subsection{Erori gasite in articol}
\subsection{Verificarea rezultatelor}
\subsection{Publicarea articolului}

%
% ---- Bibliography ----
%
%\begin{thebibliography}{5}
%
\bibliographystyle{splncs}
\bibliography{llncs}

%\end{thebibliography}

\end{document}
